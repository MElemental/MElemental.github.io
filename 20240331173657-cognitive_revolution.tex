% Created 2024-03-31 Sun 18:07
% Intended LaTeX compiler: pdflatex
\documentclass[11pt]{article}
\usepackage[utf8]{inputenc}
\usepackage[T1]{fontenc}
\usepackage{graphicx}
\usepackage{longtable}
\usepackage{wrapfig}
\usepackage{rotating}
\usepackage[normalem]{ulem}
\usepackage{amsmath}
\usepackage{amssymb}
\usepackage{capt-of}
\usepackage{hyperref}
\bibliographystyle{plain}
\author{Rudolf Jovero}
\date{\today}
\title{Cognitive Revolution}
\hypersetup{
 pdfauthor={Rudolf Jovero},
 pdftitle={Cognitive Revolution},
 pdfkeywords={},
 pdfsubject={},
 pdfcreator={Emacs 29.2 (Org mode 9.7)}, 
 pdflang={English}}
\usepackage{biblatex}
\addbibresource{~/Documents/org-roam/references/zoteromain.bib}
\begin{document}

\maketitle
\tableofcontents

The cognitive revolution refers to the shift from \url{roam:Behaviorism} and \url{roam:structuralism} in psychology and linguistics, into a paradigm of cognition.
The cognitive revolution was largely influences by the fields of information theory and the new field of computer science from 1940 to 1956.
\autocite{frankishCambridgeHandbookCognitive2012}
\end{document}
